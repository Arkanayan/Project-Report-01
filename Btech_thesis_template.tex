

\documentclass[11pt, a4paper]{report}

\usepackage[utf8]{inputenc}


\usepackage{hyperref}


\usepackage[a4paper, total={6in, 8in}]{geometry}



\usepackage{ragged2e} %--For text alignment

\newenvironment{frontmatter}{}{\maketitle}

\usepackage{longtable}

\usepackage{graphicx}
\usepackage{rotating}
\usepackage{amssymb,amsmath}
%% The lineno packages adds line numbers. Start line numbering with
%% \begin{linenumbers}, end it with \end{linenumbers}. Or switch it on
%% for the whole article with \linenumbers after \end{frontmatter}.
%\usepackage{lineno}
\newtheorem{theorem}{Definition}
\usepackage{subfigure}
\usepackage{lscape}
\usepackage{etoolbox}
\usepackage{xstring}
\usepackage{textcomp}

%\usepackage{datetime}

%\usepackage{fancyhdr}
% 
%\pagestyle{fancy}
%\fancyhf{}
%\rhead{Put your tit}
%\lhead{Title}
%\rfoot{Page \thepage}

% Enter Details about your project here
\newcommand{\ProjectTitle}{Regression Analysis using Swarm Intelligence}
\newcommand{\AuthorName}{Arkanayan Shit}
\newcommand{\SupervisorName}{Tapas Si}
% Leave unchanged if you don't have co-supervisor
\newcommand{\CoSupervisorName}{Asim Kumar Mahadani}
\newcommand{\HODName}{Aloke Roy}

\begin{document}

\begin{titlepage}
    \begin{center}
        \vspace*{0.01cm}
        \huge
        \textbf{\ProjectTitle}
        
        
        \vspace{1.5cm}
        
        Submitted by\\
        \textbf{\AuthorName}
         
         University Roll No.10500113001         
        
        \vfill
        
        A project report  submitted in partial fulfillment for the degree of
       Bachelor of Technology in Computer Science \& Engineering
        
        \vspace{0.8cm}
        
        \includegraphics[width=0.4\textwidth]{buie_logo.jpg}
        
        Department Computer Science \& Engineering\\
        Bankura Unnayani Institute of Engineering\\
        Bankura - 722 146, West Bengal, India\\
       
       {January}{\hspace*{0.5cm}}{2017}
       %\date{\displaydate{date}}
        
    \end{center}
\end{titlepage}
\newpage

\pagenumbering{gobble} 

\begin{center}
\vspace*{8.5cm}
\LARGE
\textit{Dedicated to my parents}

\end{center}



\newpage
{\it ``We are what our thoughts have made us; so take care about what you think. Words are secondary. Thoughts live; they travel far.''}
\begin{flushright}
- Swami Vivekananda
\end{flushright}


{\it ``Take up one idea. Make that one idea your life - think of it, dream of it, live on that idea. Let the brain, muscles, nerves, every part of your body, be full of that idea, and just leave every other idea alone. This is the way to success.''\\}

\begin{flushright}
- Swami Vivekananda
\end{flushright}

{\it ``When an idea exclusively occupies the mind, it is transformed into an actual physical or mental state.''}
\begin{flushright}
- Swami Vivekananda
\end{flushright}

{\it ``You have to dream before your dreams can come true.''}
\begin{flushright}
-Dr. A. P. J. Abdul Kalam
\end{flushright}


{\it ``Dream is not that which you see while sleeping it is something that does not let you sleep.''}
\begin{flushright}
-Dr. A. P. J. Abdul Kalam
\end{flushright}

\newpage


 \begin{center}
        \vspace*{1cm}
         \LARGE
        \textbf{Certificate of Approval}
        
        \vspace{1.5cm}
 \justify      
		The forgoing project report is hereby approved as a creditable study of Technological subject carried out and presented in a manner satisfactory to warrant its acceptance as a prerequisite with degree for which it has been submitted. It is to be understood that by this approval, the undersigned do not necessarily endorse or approve any statement made, opinion expressed or conclusion drawn there in but approve the thesis only for the purpose for which it has been submitted.        
        
        \vspace{1.5cm}
        
        Board of Examiners
        
        \begin{itemize}
        \item 
        \item
        \item
        \item
        \end{itemize}
        
        \vfill
        
       
        
        
        
    \end{center}

\newpage


\begin{center}
        \vspace*{1cm}
         \LARGE
        \textbf{Declaration}
        
        \vspace{1.5cm}
 \justify      
		I hereby declare that this submission is my own work and that, to the best of my knowledge and belief, it contains no material previously published or written by another person nor material which has been accepted for the award of any other degrees or diplomas of the university or other institutes of higher learning, except which due acknowledgment has been made in this text.         
        
        \vspace{5.5cm}
        
  \begin{flushleft}
   ...............................................\\
      \hspace{2.5cm}Signature
\end{flushleft}        
      
 
        
        \vfill
        
       
        
        
        
    \end{center}



\newpage
\begin{center}
        \vspace*{1cm}
         \LARGE
        \textbf{Certificate}
        
        \vspace{1.5cm}
 \justify      
		This is certified that the work contained in this report entitled, \ProjectTitle, by \AuthorName, has been carried out under the supervision of the undersign and this work has not been submitted elsewhere for any other degree.       
        
        \vspace{2.5cm}
 \small       
  \begin{flushleft}
(Signature of the Supervisor)\\
Mr. \SupervisorName, Assistant Professor, Department of CSE\\
Bankura Unnayani Institute of Engineering\\
Bankura, West Bengal

\vspace{1.5cm}

% If you have co supervisor, then only it will show
\IfStrEq{\CoSupervisorName}{CoSupervisorName} {

}{
	(Signature of the Co-supervisor)\\
	Mr. \CoSupervisorName, Assistant Professor, Department of CSE\\
	Bankura Unnayani Institute of Engineering\\
	Bankura, West Bengal
}


\vspace{1.5cm}

(Signature of the HoD)\\
Mr. \HODName, Associate Professor, Department of CSE\\
Bankura Unnayani Institute of Engineering\\
Bankura, West Bengal


\end{flushleft}        
      
 
        
        \vfill
        
       
        
        
        
    \end{center}
\newpage

\begin{center}
        \vspace*{1cm}
         \LARGE
        \textbf{Acknowledgments}
        
        \vspace{1.5cm}
 \justify      
		I hereby wish to express my sincere gratitude and respect to Assistant Prof. \SupervisorName, Dept. of CSE, Bankura Unnayani Institute Of Engineering, Bankura under whom I had proud privilege to work. His valuable guidance and encouragement have really led me to the path of completion of this project. Any amount of thanks would not be enough for the valuable guidance of my supervisor. 
I would also like to thank all the faculty member of CSE dept. for their devoted help. I also cordially thank all laboratory assistants for their cooperation.
Finally, I would like to pen down my gratitude towards my family members for their continuous support and encouragement. It would have not been possible to complete my work without their support.
   
        
     
        
        \vfill
        
       
        
        
        
    \end{center}
    


\newpage
\vspace*{1cm}
\begin{center}
\LARGE
\textbf{Abstract}
\end{center}

\justify
Regression analysis is a form of predictive modelling technique which investigates the relationship between a dependent (target) and independent variable (s) (predictor). A large part of Regression analysis deals with optimization problems. Optimization is finding the optimum (maximum or minimum) value of a co-efficients for which a function has minimum difference between predicted value and real value. Some optimization algorithms are Gradient Descent, Particle Swarm Optimization, Differential Evolution, Fireworks algorithm.

 
 \vspace*{1cm}
 \smallskip
\noindent \textbf{Keywords.} Regression analysis, Gradient Descent, Particle Swarm Optimization, Differential Evolution, Fireworks algorithm

 

\newpage
%\addcontentsline{toc}{chapter}{*}
\tableofcontents



\listoffigures


\newpage

\listoftables

\newpage
%\setcounter{page}{1}
\pagenumbering{arabic}
\chapter{Introduction}
%\justify
Regression analysis is a form of predictive modelling technique which investigates the relationship between a dependent (target) and independent variable (s) (predictor). This technique is used for forecasting, time series modelling and finding the causal effect relationship between the variables. For example, relationship between rash driving and number of road accidents by a driver is best studied through regression. 
\\

Regression analysis is an important tool for modelling and analyzing data. Here, we fit a curve / line to the data points, in such a manner that the differences between the distances of data points from the curve or line is minimized.  \cite{desc:RegressionAnalysis}
\\
\begin{figure}[!bth]
	\center
	\includegraphics[scale=0.6]{images/Linear_regression}
	\caption[Regression Analysis]{Regression Analysis \cite{wiki:LinearRegression}}
	\label{fig:regressionanalysis}

\end{figure}
	

\newpage

\chapter{Background Theory}

%{\LARGE Types of Regression Techniques}

\section*{Types of Regression Techniques}

\section{Linear Regression}

It is one of the most widely known modeling technique. Linear regression is usually among the first few topics which people pick while learning predictive modeling. In this technique, the dependent variable is continuous, independent variable(s) can be continuous or discrete, and nature of regression line is linear. \\

Linear Regression establishes a relationship between dependent variable (Y) and one or more independent variables (X) using a best fit straight line (also known as regression line). \\

It is represented by an equation $ Y=a+bX + e $ , where a is intercept, b is slope of the line and e is error term. This equation can be used to predict the value of target variable based on given predictor variable(s).

\begin{figure}[!bth]
	\center
	\includegraphics[scale=0.4]{images/Linear_regression}
	\caption[Linear Regression]{Linear Regression \cite{wiki:LinearRegression}}
	\label{fig:linearRegression}
\end{figure}

The difference between simple linear regression and multiple linear regression is that, multiple linear regression has (\textgreater1) independent variables, whereas simple linear regression has only 1 independent variable. \cite{desc:RegressionAnalysis} 


\section{Logistic Regression}

Logistic regression is used to find the probability of event=Success and event=Failure. We should use logistic regression when the dependent variable is binary (0/ 1, True/ False, Yes/ No) in nature. Here the value of Y ranges from 0 to 1 and it can represented by following equation.


\[ odds = \frac{p}{(1-p)} = \frac{probability \ of event \ occurrence}{probability \ of \ not \ event \ occurrence}
\]
\[ \ln{odds}  = \ln{\frac{p}{(1-p)}} \]


\[ logit (p) = \ln{\frac{p}{(1-p)}} \]

Above, \textit{p} is the probability of presence of the characteristic of interest. A question that you should ask here is "why have we used log in the equation?". \\

Since we are working here with a binomial distribution (dependent variable), we need to choose a link function which is best suited for this distribution. And, it is \href{https://en.wikipedia.org/wiki/Logistic_function}{\textit{logit}} 
function. In the equation above, the parameters are chosen to maximize the likelihood of observing the sample values rather than minimizing the sum of squared errors (like in ordinary regression). \cite{desc:RegressionAnalysis}

\begin{figure}[!bth]
	\center
	\includegraphics[scale=0.5]{images/Logistic_Regression.png}
	\caption[Logistic Regression]{Logistic Regression \cite{desc:RegressionAnalysis}}
	\label{fig:logisticRegression}
\end{figure}


\section{Polynomial Regression}

A regression equation is a polynomial regression equation if the power of independent variable is more than 1. The equation below represents a polynomial equation:

\[ y=a+bx^2 \]

In this regression technique, the best fit line is not a straight line. It is rather a curve that fits into the data points. \cite{desc:RegressionAnalysis}

\begin{figure}[!bth]
	\center
	\includegraphics[scale=0.5]{images/Polynomial.png}
	\caption[Polynomial Regression]{Polynomial Regression \cite{desc:RegressionAnalysis}}
	\label{fig:ploynomialRegression}
\end{figure}

\subsection{Important Points}

\begin{itemize}
	\item {While there might be a temptation to fit a higher degree polynomial to get lower error, this can result in over-fitting. Always plot the relationships to see the fit and focus on making sure that the curve fits the nature of the problem. Here is an example of how plotting can help: 
%		\begin{itemize}
				\begin{figure}[!bth]
				\center
				\includegraphics[scale=0.6]{images/underfitting-overfitting.png}
				\caption[Underfitting \& Overfitting]{Underfitting \& Overfitting \cite{desc:RegressionAnalysis}}
				\label{fig:underfittingOverfitting}
			\end{figure}
%		\end{itemize}

	}
	\item Especially look out for curve towards the ends and see whether those shapes and trends make sense. Higher polynomials can end up producing wierd results on extrapolation. \cite{desc:RegressionAnalysis}
\end{itemize}

\section{Applications of Regression Analysis}

Regression analysis estimates the relationship between two or more variables. Let’s understand this with an easy example: \\

Let\textquotesingle s say, you want to estimate growth in sales of a company based on current economic conditions. You have the recent company data which indicates that the growth in sales is around two and a half times the growth in the economy. Using this insight, we can predict future sales of the company based on current \& past information. \\

There are multiple benefits of using regression analysis. They are as follows:

\begin{itemize}
	\item It indicates the significant relationships between dependent variable and independent variable.
	
	\item It indicates the strength of impact of multiple independent variables on a dependent variable.
\end{itemize}

\subsubsection{Predicting the Future}
The most common use of regression in business is to predict events that have yet to occur. Demand analysis, for example, predicts how many units consumers will purchase. Many other key parameters other than demand are dependent variables in regression models, however. Predicting the number of shoppers who will pass in front of a particular billboard or the number of viewers who will watch the Super Bowl may help management assess what to pay for an advertisement. Insurance companies heavily rely on regression analysis to estimate how many policy holders will be involved in accidents or be victims of burglaries, for example.\cite{desc:RegressionApplication}

\subsubsection{Optimization}
Another key use of regression models is the optimization of business processes. A factory manager might, for example, build a model to understand the relationship between oven temperature and the shelf life of the cookies baked in those ovens. A company operating a call center may wish to know the relationship between wait times of callers and number of complaints. A fundamental driver of enhanced productivity in business and rapid economic advancement around the globe during the 20th century was the frequent use of statistical tools in manufacturing as well as service industries. Today, managers considers regression an indispensable tool.\cite{desc:RegressionApplication}


\newpage
{\bf Mathematical Equations}\\

You can write your mathematical equation as like in Eq.(\ref{eq:1}).

\begin{equation}
\label{eq:1}
X_{i}^{2}+Y_{j}^{2}=Z_{k}^{2}
\end{equation}


\newpage

\chapter{Review}

\newpage
\chapter{Materials \& Methods}
\section{Example of Table}



\begin{table}[h]
\centering
\caption{Table to test captions and labels}
\label{table:1}
\begin{tabular}{ |c| c| c| }
\hline
 cell1 & cell2 & cell3 \\ 
 \hline
 cell4 & cell5 & cell6 \\ 
 \hline 
 cell7 & cell8 & cell9   \\ 
 \hline
\end{tabular}

\end{table}




The table \ref{table:2} is an example of referenced \LaTeX elements.
\begin{sidewaystable}[h]
%\begin{table}[]
\centering
\caption{Table to test captions and labels}
\label{table:2}
\begin{tabular}{ |c| c| c| }
\hline
 cell1 & cell2 & cell3 \\ 
 \hline
 cell4 & cell5 & cell6 \\ 
 \hline 
 cell7 & cell8 & cell9   \\ 
 \hline
\end{tabular}

%\end{table}
\end{sidewaystable}




{\bf Multi-page tables}\\

If you have to insert a very long table, which takes up two or more pages in your document, use the longtable package. First, add to the preamble the line

\begin{verbatim}
\usepackage{longtable}
\end{verbatim}


\begin{longtable}[c]{| c | c |}
 \caption{Long table caption.\label{long}}\\
 
 \hline
 \multicolumn{2}{| c |}{Begin of Table}\\
 \hline
 Something & something else\\
 \hline
 \endfirsthead
 
 \hline
 \multicolumn{2}{|c|}{Continuation of Table \ref{long}}\\
 \hline
 Something & something else\\
 \hline
 \endhead
 
 \hline
 \endfoot
 
 \hline
 \multicolumn{2}{| c |}{End of Table}\\
 \hline\hline
 \endlastfoot
 
 Lots of lines & like this\\
 Lots of lines & like this\\
 Lots of lines & like this\\
 Lots of lines & like this\\
 Lots of lines & like this\\
 Lots of lines & like this\\
 Lots of lines & like this\\
 Lots of lines & like this\\
 ...
 Lots of lines & like this\\
 \end{longtable}

%The table \ref{table:3} is an example of referenced \LaTeX elements in landscape.
%
%\begin{landscape}
%
%\begin{table}[bth]
%\begin{center}
%\caption{Table to test captions and labels}
%\label{table:3}
%\begin{tabular}{ |c| c| c| }
%\hline
% cell1 & cell2 & cell3 \\ 
% \hline
% cell4 & cell5 & cell6 \\ 
% \hline 
% cell7 & cell8 & cell9   \\ 
% \hline
%\end{tabular}
%
%\end{center}
%\end{table}
%
%\end{landscape}

\newpage

\chapter{Results \& Discussion}

The graph is given in Fig.~\ref{fig:logo}.

\begin{figure}[!bth]
\center
\label{fig:logo}
\includegraphics[scale=0.6]{f1.png}
\caption{Graph}
\end{figure}


If you need to place multiple images all together, use this...


\begin{figure}[!h]
  \centering
   \begin{tabular}{l l l}
  \subfigure[\label{db1mri1}]{\includegraphics[width=1.4in,height=1.2in]{f1.png}}   &
   
  \subfigure[\label{db1mri2}]{\includegraphics[width=1.4in,height=1.2in]{f2.png}}   &
 
  \subfigure[]{\includegraphics[width=1.4in,height=1.2in]{f3.png}}  \\

  \subfigure[]{\includegraphics[width=1.4in,height=1.2in]{f4.png}}   &
 
  \subfigure[]{\includegraphics[width=1.4in,height=1.2in]{f5.png}}   &

  \subfigure[]{\includegraphics[width=1.4in,height=1.2in]{f6.png}}   \\
   \end{tabular}   
  \caption{put the caption}.
 \label{fig:db1}
\end{figure}



\newpage

\chapter{Conclusions}

\newpage

\chapter{Future Works}
Here you can outline the future scopes of your work..........
\newpage

\chapter*{List of Publications}
\begin{enumerate}
\item Author's name, Title, Conference, date, place, year, pp.~xx-xx.
\item Author's name, Title, Journal, Vol~X, No.~X, Year, pp.~xx-xx.
\end{enumerate}
\newpage
\begin{thebibliography}{50}
	
\bibitem{desc:RegressionAnalysis}
\url{https://www.analyticsvidhya.com/blog/2015/08/comprehensive-guide-regression}

\bibitem{wiki:LinearRegression}
\url{https://en.wikipedia.org/wiki/Linear_regression}

\bibitem{desc:RegressionApplication}
\url{http://smallbusiness.chron.com/application-regression-analysis-business-77200.html}

\end{thebibliography}


\end{document}
